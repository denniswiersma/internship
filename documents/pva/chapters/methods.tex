\chapter{Methods}
This project can be divided into several phases, which will be discussed below.

\section{Literature}
The techniques used by this research group will be read up on to get a sense of what they are working on.
Additionally, some elements in their work's biological context need revision to have a good understanding of the subject matter.

Independent Component Analysis (ICA) is at the centre of the research performed by this group and is, therefore, one of the most important techniques to understand for this project.
Although an mRNA expression profile is the starting point of the analysis process, this isn’t the primary dataset that will be utilised.
Instead, a post-ICA mixing matrix will be employed.
This means a product of ICA is the starting point of this project, making it all the more important to understand what goes in, what comes out, and what happens when performing ICA.

\section{Exploratory Data Analysis}
To gain familiarity with the data at hand and preprocess said data, an Exploratory Data Analysis (EDA) will be performed.
The exact methods to be employed for this process will reveal themselves as more knowledge is gained, but there are a few steps that can be performed in any case:
\begin{itemize}
    \item Calculate basic statistics for important features.
    \item Perform hypothesis tests on (select subsets of) the data in order to gain insight into the relations between variables.
    \item Produce a heatmap to visualise relations between variables, as is typical in the analysis of mRNA profiles.
    \item Filter the data for missing values.
\end{itemize}
Exactly how these steps manifest themselves and the results will guide any follow-up analysis.

\section{Machine Learning}
The main objective of this project is to produce a machine learning algorithm which can accurately predict what cancer type belongs to a given mRNA expression profile.
Therefore, it is one of the most essential and extended phases and should be performed carefully.

As mentioned in the previous chapter, incomplete annotations form a bottleneck in the analysis of mRNA profiles.
Additionally, a straightforward method like GBA does not suffice.
Therefore, an algorithm that can correctly annotate mRNA expression profiles may provide meaningful value.

Before commencing this project, ICA has been performed on the mRNA profiles.
ICA outputs data that may be separated into two different datasets, of which one is of interest for this project.
This dataset is a mixing matrix containing weights representing the activity of Transcriptional Components (TCs) in biopsy samples.
This mixing matrix will be used to train the machine learning algorithm.
The samples can be matched to their respective cancer type, providing a meaningful categorical variable that may serve as the target variable.

The problem lends itself well to a classic method like a decision tree or a random forest.
More elaborate methods may be employed, but an elegant, less involved process would be preferable in the interest of time.
However, other methods may be explored if trees or forests yield insufficient performance.
The model should perform to a level that is satisfactory to the point where it may be used in researchers' analysis pipelines.

\section{Webapplication}
A secondary objective for this project is to develop a web application around the trained machine learning model.
The goal is to let the user supply unannotated data, have the model provide annotations and a confidence level for the prediction, and return the data to the user.

The application will be written in Python.
Python has two dominant web frameworks: Flask and Django.
Since the web application is a secondary objective, the choice of framework will be deferred to when this phase is reached.

HTML combined with CSS is the obvious way to go on the front.
While HTML does not confront us with more decisions, CSS does.
It may be wise to choose a CSS flavour that decreases the development time needed instead of handwriting CSS.
There exist many wrappers around CSS, and for the same reason as mentioned above, a decision here will be deferred to a later phase.

\section{Workflow}
To keep things running smoothly, effort must be put into the system rather than the work itself.

Code will be collected and presented in a git repository hosted on GitHub.
This repository should provide ample documentation on both methods employed during development and instructions on deploying the final product.
To keep the repository organised, the Git flow method of branching will be used.

Tracking todos will be done using GitHub’s project system.
This will provide integration with the repository, making issues easier to manage.

Lastly, if time allows, a DOCKERFILE will be written to containerise the application, allowing painless deployment.
