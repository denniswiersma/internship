%&pdflatex
\chapter{Introduction}

\section{Objective}
[Obviously, this should be much larger]

In a significant amount of publicly available mRNA expression profiles (cancer type) annotations might be missing. [this probably needs a citation]
Additionally, mistakes are easily made when annotating this data by humans. [this too]
So, what if we were to learn a machine to do this annotation work for us?
If the ML model is robust enough, we might even be able to detect wrongly annotated data.

The model is trained on mRNA expression profile (TCGA) data.
If it also performs well on microarray data (GPL570), [other thing] is successful at getting rid of platform and batch effects.
% Someone else in the research group has been working on harmonising transcriptional data (e.g., mRNA expression profiles and microarray data).
% If the model also performs well on a (harmonised) dataset from microarray data, this would show that the model is good/robust but also show that the harmonisation works well.

The model isn't just trained on raw mRNA expression profiles.
Instead, ICA is performed on the data which results in, among other things, a mixing matrix consisting of samples and their weights in different Transcriptional Components.
More on that in data acquisition of methods.
